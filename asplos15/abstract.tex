\documentclass[paper.tex]{subfiles}
\begin{document}

\begin{abstract}

  From scientific computing to financial forecasting, many
  applications depend on hardware-supported floating point
  arithmetic. Developers of these applications typically use floating
  point to efficiently approximate computation over real numbers.
  Unfortunately, this approximation introduces rounding error, which
  can compound to produce completely incorrect results.  If the
  largest hardware-supported precision cannot produce sufficiently
  accurate results, developers may be forced to incur the costs of
  simulating arbitrary precision floating point in software.  While
  the numerical methods literature provides techniques to mitigate
  rounding error without increasing precision, applying these
  techniques requires deep expertise and substantial manual effort.

  We introduce \casio, an \textit{automated} technique to mitigate
  rounding error without resorting to software floating point.  Given
  a real number formula, \casio combines three novel techniques to
  search for a floating point implementation that improves accuracy:
  (1) an approach to estimating error and its causes which is tailored
  to managing the high branching factor of \casio's search; (2) an
  approach to detecting regimes where error behavior changes which
  allows \casio to combine multiple programs and reduce overall error;
  (3) a technique that iteratively applies local, domain specific
  rewrites, from which traditional error-reducing program
  transformations fall out naturally. We evaluate \casio on benchmarks
  drawn from numerical methods textbooks, mathematical libraries, and
  recent scientific articles, demonstrating that \casio substantially
  reduces error while imposing modest overhead.

\end{abstract}

% Furthermore, such incorrect results are difficult to detect and
% debug.

% ideal real number computation

% The numerical methods literature provides techniques to improve the
% numerical accuracy of real computations approximated in floating
% point.  Unfortunately, applying these techniques requires
% substantial expertise and manual effort: analysis + transformation.
% As more and more scientists write numerical programs, the lack of
% accuracy becomes increasingly troubling.  These non-expert
% programmers usually lack the training and time necessary to employ
% numerical methods in their code, and often may not even be aware of
% the subtle challenges this domain presents.

\end{document}