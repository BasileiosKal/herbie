\documentclass[paper.tex]{subfiles}
\begin{document}

\begin{abstract}

  Many algorithms are designed to compute over real numbers, but
  performance considerations force practical implementations to
  approximate their computations using hardware-supported, finite
  precision floating point.  Unfortunately, finite precision leads to
  rounding error, which can compound to produce completely incorrect
  results.  If the largest hardware-supported precision cannot produce
  sufficiently accurate results, developers are typically forced to
  incur the costs of simulating arbitrary precision floating point in
  software.  While the numerical methods literature provides
  techniques to mitigate rounding error without increasing precision,
  manually applying these techniques requires substantial expertise.

  We introduce \casio, an \textit{automated} technique to mitigate
  rounding error without resorting to software floating point.  \casio
  applies numerical methods techniques to transform floating point
  computations to more accurately approximate their ideal real-number
  computations.  \casio performs a heuristic search over program
  transformations, composing simple, local rewrites to automatically
  achieve results similar to traditional error-reducing
  transformations.  This search is directed by methods to estimate
  rounding error and its sources.  \casio automatically detects
  regimes where error behavior differs, and combines multiple programs
  to reduce error across all regimes.  We evaluate \casio on
  microbenchmarks drawn from numerical methods textbooks, mathematical
  libraries, and recent scientific articles. % \todo{results}

\end{abstract}

% Furthermore, such incorrect results are difficult to
%   detect and debug. 

  % The numerical methods literature provides techniques to mitigate
  % rounding error.  Unfortunately, manually applying these techniques
  % requires substantial expertise.  In practice, developers often
  % address rounding error by increasing precision at the cost of
  % performance and memory usage.  However, once the largest
  % hardware-supported precision has been reached, the developer
  % typically has few tools to reduce rounding error while keeping all
  % floating point operations in hardware.



%   Floating point computation is a well known approach to approximating real
% computation in the bounded context of practical hardware.  Recent advances
% provide techniques to trade-off accuracy for improved speed and power usage
% at the cost of precision.  However, this work is often carried out under
% the assumption that the original floating program is the ``ground truth'',
% that is, that it is already an acceptable approximation of the original
% real computation the programmer had in mind.  Unfortunately, approximating
% real computation in floating point presents many subtle challenges which,
% if not accounted for, can yield arbitrarily large errors.  This limits the
% value of recent advances improving the runtime and power efficiency of such
% computations: the wrong answer is simply arrived at more quickly using
% fewer joules (watts? ugh).

% The field of numerical methods provides a rich set of techniques for
% improving the numerical accuracy and precision of real computations
% approximated in floating point.  Unfortunately, applying these techniques
% has traditionally demanded extensive training and expensive manual analysis
% and optimization of numerical programs.  As more and more scientists write
% numerical programs, this lack of accuracy becomes increasingly troubling.
% These non-expert programmers usually lack the training and time necessary
% to employ numerical methods in their code, and often may not even be aware
% of the subtle challenges this domain presents.

% We present a new tool, \casio, which synthesizes floating point programs
% which better approximate the real computations programmers have in mind.
% Casio works by performing a heuristic search over the space of programs
% equivalent to the input program over the reals.  Each step of the search
% attempts to apply identity transformations over the reals that have been
% shown valuable in the numerical methods literature.  Casio also searches
% for alternate representations of programs that may be challenging for even
% for numerical methods experts to develop, for example, special casing
% variants of the program to input domains with better precision for that
% variant.

% We evaluate \casio on a number of examples drawn from numerical methods
% textbooks, scientific journal articles, and math libraries.  We find that
% \casio is often able to improve the precision of these examples
% considerably.

\end{document}
