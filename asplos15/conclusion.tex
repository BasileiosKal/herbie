\documentclass[paper.tex]{subfiles}
\begin{document}

\section{Conclusion}
\label{sec:conclusion}

Approximations of real computations through floating point hardware
  are necessary in scientific computing simulations, data analysis,
  and a variety of other applications used by scientists,
  engineers, and mathmeticians.
These approximations, however, accumulate rounding error,
  and can produce wholly inaccurate results.
While increasing precision can provide additional accuracy to a point,
  slow software-simulated arbitrary precision floating point
  was often the only method available to developers to increase accuracy
  to the desired level,
  unless they happen to be deeply knowledgable in floating point behavior,
  and numerical methods techniques to deal with it.

Through \casio, developers can now easily improve the accuracy
  of their floating point calculations
  without any special knowledge of the intricacies of floating point.
\casio automatically improve the accuracy of its input
  by searching for real-equivilent, error reduced implementations,
  through estimation of rounding error and its sources,
  and analysis of complementary implementations across the input space.
\casio is often able to significantly improve
  benchmarks taken from Numerical Methods for Scientists and Engineers,
  as well as real world formulas.

\casio is the first tool
  which can improve the accuracy of floating point programs
  by rearranging their computations
  without requiring input from a numerical methods expert.
\casio also defines the first correctness criteria for
  this class of transformation,
  and guides it's search through a sampling based approach.
Furthermore, the \casio infrastructure provides
  a foundation for others to build upon
  and explore alternate techniques to mitigate rounding error,
  improve numerical stability,
  and develop interactive numerical tools.

\end{document}
