\documentclass[paper.tex]{subfiles}
\begin{document}

\section{Related Work}
\label{sec:relatedwork}

\casio builds upon decades of work in numerical analysis,
  along with recent advances in the analysis
  of floating point computations.

\paragraph{Standards}
Floating point computation is defined in the IEEE 754 standard,
  originally in 754-1985~\cite{ieee85-standard}
  and most recently 754-2008~\cite{ieee08-standard}.
This standard specifies the sizes, binary layout, and rounding modes
  for floating point numbers of various sizes,
  along with the required behavior of primitive operations.
However, programming languages often
  do not require compilers to follow the standard exactly.
For example, C~\cite{c99} allows intermediate results
  to be computed in extended precision,
  and only truncated when stored into memory~\cite{toplas08-pitfalls-verifying}.
This makes verification and analysis difficult
  for floating point computations.

\paragraph{Numerical analysis}

Numerical analysis studies how to use computers
  to perform mathematical computations.
This body of work includes a vast literature
  on how to evaluate mathematical functions.
The technique of rearranging formulas into more-accurate forms
  appears in D.~Goldberg's survey~\cite{acm91-every-scientist},
  as well as common textbooks~\cite{book87-nmse}.
It is extensively discussed by W.~Kahan in the context
  of several simple geometric problems in~\cite{berkeley00-needle-like}.
Such rearrangement is largely discussed as a manual activity,
  done by a numerical methods expert.
\casio uses the techniques invented in this literature,
  but acts automatically, avoiding the need for an expert.

It is difficult to know the precision of intermediate results
  necessary to accurately evaluate a function,
  a problem known as the table maker's dilemma~\cite{ensl03-table-maker}.
Recent work on this problem allowed the creation of MPFR,
  an arbitrary-precision floating point library
  which correctly rounds exponential, trigonometric, and other basic functions~\cite{acm07-mpfr}.
However, composing these functions can still lead to error.
\casio uses MPFR internally to determine
  the exact output of a computation,
  to compare candidate programs for accuracy.

\paragraph{Verification}
Recent work in the program verification community
  has allowed formal verification of numerical programs.
The triangle area computation discussed by Kahan, for example,
  was proven to be accurate by S. Boldo in 2013~\cite{arith13-area-triangle}
  in the interactive proof assistant Coq.
Larger computations have also be proven accurate,
  such as programs for computing discriminants~\cite{ieee09-discriminant-proven}
  or for solving simple partial differential equations~\cite{jar13-wave-equation-prove}.
These proofs were all done in interactive theorem provers.
Automatic proofs have been explored in Rosa~\cite{popl14-rosa},
  which provided a language for specifying accuracy requirements
  and applied SMT solvers to automatically prove
  that programs attained them.
\casio does not verify the accuracy of programs it produces;
  the existing tools could be used to do that.
Ariadne~\cite{popl13-ariadne} also applied SMT solvers
  to detect inputs which could cause floating point overflow.
However, their technique could not conclusively prove
  the absence of floating point exceptions,
  as it approximates floating point arithmetic by real arithmetic.
\casio focuses on the problem of floating point accuracy,
  instead of overflow.

\paragraph{STOKE}
The Stoke super-optimizer supports
  optimizing floating point programs while guaranteeing
  that the resulting accuracy is acceptable~\cite{pldi14-stoke}.
Stoke uses a stochastic search over assembly programs
  to find faster implementations of a given function.
When optimizing a program that uses floating point operations,
  Stoke allows the user to specify a maximum allowable error
  and produces programs whose output is equivalent to the original,
  up to this allowable error.
Stoke measures error compared to the program given as input;
  thus, it assumes the input program accurately represents
  the computation the program intended to perform.
If the input program is inaccurate, Stoke can at best guarantee
  that the accuracy of the optimized result is not much worse.
On the other hand, \casio attempts to improve the input program,
  so that it better represents
  the real number formula that it implements,
  a different technique is needed.
We suspect that Stoke and \casio may complement each other,
  with \casio improving the accuracy of a program,
  so that its error (relative to a real number computation) is acceptable,
  and Stoke then optimizing the program within the given error bounds.

\end{document}
