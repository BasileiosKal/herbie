\documentclass[paper.tex]{subfiles}
\begin{document}

\section{Future Work}
\label{sec:futurework}

We intend to improve \casio's search for accuracy-improving
transformations and also work toward automating additional numerical
analysis techniques, e.g. supporting complex number and matrix
computations, reducing error accumulation in loops, and analyzing
numerical stability.  We believe that \casio's insight of using
real-equivalence to enable sampling techniques that avoid overfitting
can help in all these areas.

Eventually, we hope to integrate \casio into a compiler pass, allowing
it to run transparently as part of a compilation pipeline.  This would
make \casio significantly easier to use, as it would obviate the need
to isolate and input floating point computations.  Since compilers
must optimize for speed as well as numerical accuracy, we would like
to allow the user to specify a tradeoff between speed and accuracy.

We would also like to explore allowing \casio to output programs that
are not real-equivalent to the original program.  This would enable
using Taylor expansions and interpolation to improve the accuracy of
input programs, but will require error analysis techniques less
dependent on sampling.  While real-equivalent guarantees that \casio
preserves the programmer's intent, some programs cannot be improved at
all, or can be improved more efficiently, by making unsound
transformations.  This may also lead to further explorations of
precision vs. performance tradeoffs for floating point.

\end{document}
