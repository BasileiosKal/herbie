\documentclass[paper.tex]{subfiles}
\begin{document}

\section{Future Work}
\label{sec:futurework}

As this paper shows, \casio is a useful tool
  for mitigating floating point rounding error.
We hope to continue to improve \casio,
  making it a more useful tool for scientists and engineers,
  and working toward automating numerical analysis
  for use by non-experts.

We hope to integrate \casio
  into a compiler pass, allowing it run transparently
  as part of a compilation pipeline.
This would make \casio significantly easier to use,
  as it would obviate the need to isolate and input floating point computations.
Since compilers must optimize for speed as well as numerical accuracy,
  we would like to allow the user to specify a trade off
  between the speed of the resulting program and its accuracy.
We are also working to extend \casio
  to work on a broader class of programs
  by supporting complex number and matrix operations,
  and by recognizing and handling error accumulation
  across loops.

We would also like to allow \casio to output programs
  that are not real-equivalent to the original program.
This would allow using Taylor expansions and interpolation
  to improve the accuracy of input programs.
While real-equivalent guarantees that \casio preserves
  the programmer's intent,
  some programs cannot be improved at all,
  or can be improved more efficiently,
  by making unsound transformations.

\end{document}
