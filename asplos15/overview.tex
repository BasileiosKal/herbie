\documentclass[paper.tex]{subfiles}
\begin{document}

\section{Overview}
\label{sec:overview}

\todo{use quadratic?}

We introduce \casio, which \textit{automatically} improves floating
point accuracy by searching for error-reducing program
transformations.  This search is guided by techniques for estimating
rounding error and its sources, as well as mechanisms which detect
input regimes with distinct error behavior and select the best program
variant for each regime.  We evaluate \casio on examples from a
classic numerical methods textbook and demonstrate that it can
effectively reduce rounding error. We also consider \casio's broader
applicability by improving the accuracy of expressions from a
mathematical library as well as formulas from recent scientific
articles.

As a first step toward addressing these challenges, we introduce
\casio, a technique for automatically mitigating rounding error in
floating point computations.  \casio searches for error-reducing
program transformations by combining three novel techniques.  First,
\casio estimates which operations are responsible for rounding error
by sampling floating point inputs and comparing the results of
intermediate operations against accurate results computed using
arbitrary precision floating point.  Second, \casio applies various
rewrites based on the error's source, measures their individual
effects on accuracy, and chooses a subset of program variants to
continue improving.  Third, \casio detects input regimes where error
behavior changes and combines multiple program variants to reduce
overall error.  By combining these techniques, \casio improves
accuracy without requiring the programmer to learn the details of
floating point arithmetic or manually rearrange computations.

Given the subtlety floating point error, producing the minimal-error
program for a given real number formula is intractable
\footnote{\cite{} show this task is NP-hard}.  Instead, \casio
performs a heuristic search over programs, aiming for minimal error.
Instead of attempting to transform the whole program, \casio's search
proceeds by applying local program transformations.  \todo{sampling}
Each transformation corresponds to a identity of real number formulas,
but does not necessarily preserve the semantics of the program as
evaluated with floating-point operators.  \casio contains \nRewrites
rewrite rules, most describing basic arithmetic facts such as the
commutativity of addition, others demonstrating trigonometric facts
such as the sine and cosine angle addition identities.

\casio's heuristic search is organized by two core ideas: high-level
and domain-specific strategies that apply goal-directed rewrite rules;
and pursuing different programs for different parts of the input
space.  At each step of its goal directed search, \casio focuses on
expressions with high error, uses its recursive-rewriter to destruct
the operator of that expression, and them simplifies the expression.
To pursue different programs for different inputs, \casio holds on to
all potentially relevant alternatives in candidate tables, combines
incomparable candidates using regime inference, and analyzes periodic
subexpressions for periodic regime boundries.  These two core ideas
combine the better parts of brute-force and targeted search.  By
applying high-level strategies, \casio is able to explore deeply into
the search space without running into exponential slowdown.  But like
brute-force search, \casio pursues multiple alternative
transformations in parallel, selecting whichever produces the best
result.

To demonstrate the method by which
  \casio improves the precision of floating point programs,
  this section steps through \casio as run on the formula
\begin{equation}\label{eq:ex}
  \frac1{\sqrt2} \sqrt{\sqrt{x^2 + y^2} + x},
\end{equation}
  which computes the real part of $\sqrt{x + i y}$.
This forumla is inaccurate for negative $x$,
  especially if $x$ is large and $y$ is small.
Improving the precision of this formula
  involves most of the major systems;
  each of these systems is described in detail
  in Section~\ref{sec:synthesis}.

\casio begins the process of improving~\eqref{eq:ex}
  by identifying subexpressions
  which likely need transformation.
The imprecision arises from the addition $\sqrt{x^2 + y^2} + x$:
  for negative $x$, and a small $y$,
  the two summands are approximately equal in magnitude,
  leading to a cancellation.
\casio notices this same fact
  by analyzing the behavior of the program on sample inputs.
For each expression in the program,
  \casio computes the exact value of its children
  \todo{to 64 bits? on floats / doubles?}
  using arbitrary-precision arithmetic.
A \emph{locally approximate} result is then computed
  by applying a floating-point operator to exact arguments.
By comparing this locally approximate result to the exact result,
  \casio obtains a measure of how much each operator
  contributed to the expression's overall error.
For the complex square root example,
  the addition $\sqrt{x^2 + y^2} + x$ produces the most local error
  for large negative values of $x$.
This information is passed to \casio's recursive rewrite step,
  directing rewrites toward improving this particular expression.

Once problematic expressions are identified,
  local transformations are applied at that expression.
\casio uses a recursive rewrite algorithm for doing this:
  if a rewrite rule does not immediately pattern match on the expression,
  but destructs the correct operator,
  and matches the number of arguments,
  the children of the expression are recursively rewritten
  in an attempt to make the rule match.
This rewrite step produces several candidate programs
  for each expression focused upon.
For example, rewriting at the addition discussed above
  produces seven candidate programs;
  two other expressions with nonzero local error produce 17 more.
Of these, one candidate applies the identity
  $x + y = (x^2 - y^2) / (x - y)$ to the problematic addition.
This transforms \eqref{eq:ex} into
\begin{equation} \label{eq:ex2}
  \frac1{\sqrt2} \sqrt{\frac{\sqrt{x^2 + y^2}^2 - x^2}{\sqrt{x^2 + y^2} - x}}.
\end{equation}
This candidate does not have improved precision,
  but it does introduce the possibility
  of cancelling the two $x^2$ terms.

To discover this cancellation,
  \casio cannot use focusing and recursive rewrite
  because there are too many intermediate steps:
  removing the unnecessary square root squared,
  commuting the addition $x^2 + y^2$,
  and reassociating the subtraction $(y^2 + x^2) - x^2$.
Discovering such a long sequence of undirected steps
  by enumerating all sequences of rewrites
  would take too long.
Instead, \casio has a simplification pass
  specialized toward cancelling like terms.
Simplification automatically performs all the steps necessary
  to transform the candidate~\eqref{eq:ex2} into
\begin{equation} \label{eq:ex3}
  \frac1{\sqrt2} \sqrt{\frac{y^2}{\sqrt{x^2 + y^2} - x}}.
\end{equation}
This new program has no imprecision for negative $x$,
  but it is inaccurate for positive $x$.
Regime inference will now be used to combine both programs
  into one that is inaccurate for all $x$.

In this example, as in many,
  no single-regime program is inaccurate for all input values.
To compute a formula inaccurate, the data must select
  which program to use for the formula's evaluation.
\casio's regime inference system
  computes the branch condition
  that selects between several programs depending on the input data.
Regime inference creates this condition
  by considering all generated programs
  that are best on at least one sample point,
  and then using the sample points to divide each input variable
  into several regions, with one program per region,
  although some programs may cover multiple non-adjacent regions.
In the case of programs \eqref{eq:ex} and \eqref{eq:ex3},
  regime inference correctly decides
  to use \eqref{eq:ex} for positive values of $x$
  and \eqref{eq:ex3} for negative values.
The final program produced by \casio is
\[
\begin{cases}
  \frac1{\sqrt2} \sqrt{\sqrt{x^2 + y^2} + x} & \mathtt{if} \;\; x \ge 0 \\[8pt]
  \frac1{\sqrt2} \sqrt{\frac{y^2}{\sqrt{x^2 + y^2} - x}} & \mathtt{otherwise} \\
\end{cases}
\]
\end{document}
