\documentclass[paper.tex]{subfiles}
\begin{document}

doug dad soundness prevents overfitting

\section{Evaluation}
\label{sec:evaluation}

\todo{compiler flags order of magnitude speed up !!!}

We evaluate \casio on benchmarks drawn from a numerical methods
textbook.  The results demonstrate the \casio substantially improves
the accuracy of many of these expressions, without significantly
slowing execution.  In all cases, the programs produced by \casio run
orders of magnitude faster than software floating point.  In some
cases, the accuracy of the programs outputted by \casio exceeds that
of rewrites provided by numerical methods textbooks.  We also
separately evaluate our error estimation technique and our regime
inference algorithm, and describe several case studies using \casio.

\subsection{Evaluating \casio end-to-end}

\paragraph{Benchmarks}
We chose benchmarks from Hamming's \emph{Numerical Methods for
  Scientists and Engineers}~\cite{book87-nmse}, a standard
practitioner's textbook in numerical analysis%
\footnote{It is also the first result on Amazon, at the time of
  writing, for ``numerical methods''.}.  These tests measure \casio's
performance on standard examples from numerical analysis.  To avoid
sampling bias, we include every worked example and problem from
Chapter~3, which discusses manually rearranging formulas to more
accurately evaluate them---%
the same task that \casio performs automatically.  These total
twenty-nine benchmarks, of which twenty-seven had a large rounding
error%
\footnote{The other two are examples of programs which appear to need
  rearrangement, but in fact do not need it.}.

\paragraph{\casio configuration}
\casio was run on each benchmark in a standard configuration%
  \footnote{including regime inference and
    the normal number of sample points.}.
The main loop was capped at five iterations.
Figure~\ref{fig:eval-casiotime} is a cumulative density function
  for \casio's runtime on these benchmarks.
For 17 of 29 tests, \casio runs in under one minute.

\paragraph{Measuring accuracy and speed}
The accuracy and speed was measured,
  both for the original program in each benchmark,
  and for \casio's output.
Examples were compiled to C,
  which was then compiled by the GNU Compiler Collection
  with flags optimizing for performance without loss of accuracy%
  \footnote{Version $4.9.1$ of \texttt{gcc} was used, with flags
    \texttt{-march=native}, \texttt{-mtune=native},
    \texttt{-mfpmath=both}, \texttt{-O3}, and \texttt{-flto}.
    Experiments were run on an Intel Core~i5-3317U processor.}.
\todo{What about \texttt{-ffast-math}?}
Both the input and output program were run
  with single, double, and extended double (32, 64, and 80 bits)
  intermediates.
Each program was run on the same set of 1,000,000 points
  drawn randomly from the set of single-precision floating point numbers.
Results were compared with a ground truth computed via
  the MPFR arbitrary-precision library~\cite{acm07-mpfr}.
Figure~\ref{fig:eval-mpfr-bits} is a cumulative density function
  for the number of bits necessary to exactly evaluate the benchmarks.
The error of an output computed in floating-point,
  was taken to be the base-2 logarithm of the number
  of 64-bit floating point values%
  \footnote{Inputs are single precision values,
    while outputs are double precision.
    Single precision for inputs avoids handicapping implementations
    that use single precision for intermediate values.
    Double precision for outputs allows measuring error
    more accurately than single precision would.},
  between the exact and inexact answer:
  the bits in error for the approximate output, compared to the exact answer.
This error was averaged over all points for which the exact answer
  was a finite floating point value,
  resulting in a simple number which can be used for comparing implementations.
Sampling one million points
  is a very accurate measure of the average error,
  as demonstrated in Section~\ref{sec:eval-eval}.

\paragraph{Results}
Results for all benchmarks are listed in Figure~\ref{fig:evalgraph}.
These results show that \casio can successfully improve many of the
examples from NMSE.  Of the eleven benchmarks that \casio does not
improve, eight have solutions which are not real-equivalent to the
input (taylor expansions are used to recover precision near 0).  These
benchmarks are impossible for \casio to improve without supporting
unsound rewrites.

The timing measurements demonstrate that \casio does not significantly
hurt the speed of the original program.  \casio incurs an average
overhead of \todo{N}\%.  In \todo{N} of the 29 benchmarks, \casio's
error with single precision floating point is greater than the
original program's accuracy with double precision.

Average error is a coarse measure of floating point accuracy.  \casio
presents programmers with a graph of error versus input value.  These
plots display 8000 points.  Figure~\ref{fig:errorvinput} shows these
plots for \todo{1d program} and \todo{2d program}.  These examples are
representative of all benchmarks.  In all cases, average error
improved when inputs that would produce wholly incorrect results were
changed to producing relatively accurate results.

\subsection{Evaluating point evaluation}\label{sec:eval-eval}

The tests above evaluated \casio by comparing the average error of its
input and output.  Average error was computed by sampling one million
points and computing the average error across the sample.  To
demonstrate that this accurately estimates average error, we computed
average error with the number of points $N$ varying between 1 and
10,000.  For each $N$, the experiment was repeated 100 times with
different sampled points.  The standard error of the mean of these 100
repetitions, for each value of $N$, measures the accuracy of our
computation of average error.  Figure~\ref{fig:eval-eval} is a
representative result; it was generated from the the quadratic formula
example described in Section~\ref{sec:overview}, but every test case
generated a qualitatively similar result.  In each case, sampling a
thousand points gave a result with standard error of approximately
$0.1$ bits.  This demonstrates that the numbers given above are
accurate.

To ensure that our evaluation in arbitrary precision is accurate (that
is, to ensure that we use a sufficiently large number of bits to
evaluate the results accurately), we computed the exact answer for
both \casio's input and output in arbitrary precision.  In every case,
the answers were identical, showing that arbitrary precision
evaluation is accurate. Thus our rewrite are sound.

We have run \casio with sample points being drawn
  uniformly in the range $[-\|max|_{32}, \|max|_{32}]$,
  where $\|max|_{32}$ is the largest finite value
  representable in 32 bit floating point.
\casio was only able to solve five benchmarks from NMSE,
  of which two were benchmarks where no improvement is possible
  and the other three of which use periodicity analysis
  and thus use a separate point-sampling mechanism.
This suggests that the uniformity over floating point representations,
  instead of over the range of values,
  is the correct measure to use for guiding \casio's search.

\subsection{Evaluating regime inference} \label{sec:eval-regimes}

Two experiments evaluate \casio's regime inference.
The first is an end-to-end test,
  running \casio on its standard benchmarks but with regime inference turned off.
Comparing the results of \casio with and without regime inference
  measures the effectiveness of regime inference.
Regime inference adds branches to eight of the 29 benchmarks;
  Figure~\ref{fig:eval-regimes-e2e} shows
  \casio's accuracy and speed on these eight benchmarks,
  with and without regime inference.
The results show that for these test cases, regime inference
  significantly improves the accuracy of \casio's results.
However, it also makes the resulting programs slower.
Regime inference infers branches of the form $a_< < x_i < a_>$,
  for variables $x_i$.
Since points are selected randomly, these branches are
  impossible for the processor to predict,
  leading to the large performance cost.
Practical use may lead to the inferred branch being predictable;
  so, the performance cost of regime inference
  as displayed in Figure~\ref{fig:eval-regimes-e2e},
  is a worst-case cost.

A second experiment measures regime inference in isolation.
An input point $p$ is chosen
  and regimes is asked to combine the constant candidate programs
  $10$, $20$, $30$, $40$, $50$, and $60$
  to most accurately compute $(\|if|\:x < p\:\|then|\:a\:\|else|\:b)$.
  \todo{Actually, we use different functions}
Regime inference must infer a branch of the form $x < p'$,
  with the correct candidate programs on each side of this branch;
  then, the difference between the computed value $p'$ and the ground truth $p$
  measures regime inference's ability to accurately infer branch conditions.
This experiment was repeated 100 times.
In each case, regime inference inferred a branch of the correct form;
  the difference between $p$ and $p'$ averages \todo{N}\%
  (max \todo{N}\%, min \todo{N}\%).
This suggests that regime inference infers branch conditions
  very accurately.

\subsection{Wider applicability}

To test \casio's applicability beyond its standard benchmark suite,
  we gathered mathematical formulas from a variety of sources
  and tested both their numerical accuracy
  and \casio's ability to improve it.
These sources included several papers from Volume 89 of Physical Review;
  standard definitions of mathematical functions,
  such as hyperbolic trigonometric functions,
  arithmetic on complex numbers;
  and approximations to special functions like \|erf| and $\zeta$.
Of the 68 formulas gathered, we found that
  20 exhibited significant floating point inaccuracies.
For these 20 examples, \casio was able to improve 6
  with no modifications and without enlarging its database of rules.
This is yet another confirmation
  that rounding error can arise in the daily practice of scientists and engineers.
However, it is important to note that for these examples
  we have not determined if inaccuracies arise for valid inputs;
  and, for formulas \casio was unable to improve,
  whether a real-equivalent formula with lower error exists.

\subsection{Case Study: Computing Probabilities}
Many numerical programs are not library functions or textbooks examples, 
  but are instead simulations or data analysis scripts
  used by scientists, engineers, and mathematicians
  in the course of their work.
The expressions encountered in these programs are more complex
  and less structured than examples like~\eqref{eq:ex}.
\casio has demonstrated success in improving precision
  in these more-difficult cases.

A colleague of ours had difficulties with the numerical precision
  of a estimator in a machine learning task.
He needed to compute the expression
\begin{align*}
\mathsf{ans} &= \frac{p(e_-,e_+|s)}{p(e_-,e_+|t)} \\
p(e_-, e_+|x) &= (\operatorname{sig} x)^{e_+} (1 - \operatorname{sig} x)^{e_-} \\
\operatorname{sig}x &= 1 / (1 - e^{-x})
\end{align*}
  and found that the simple encoding of this formula as a program
  lead to spurious results and violated invariants in later code.
Our estimates suggest that this simple encoding produces
  seventeen bits of error, averaged over floating point values.
To avoid these problems,
  our colleague manually manipulated the expression
  into a form that seemed not to cause similar problems;
  our estimates suggest that this improved variant
  had ten bits of average error.

Upon hearing of his troubles,
  we fed the original, simple implementation to \casio.
\casio produced a program with four bits average error:
  \begin{equation*}
  \operatorname{exp}\left(e_+\ln{\frac{1+e^{-t}}{1+e^{-s}}} +
     e_-\ln{\frac{1-\frac{1}{1+e^{-s}}}
                 {1-\frac{1}{1+e^{-t}}}}\right)
  \end{equation*}
\casio obviated the need for manual algebraic manipulation,
  and produced superior results with no manual steps.
The low-error program that \casio produced was of a complexity level
  such that human analysts would be unlikely to reproduce the result. 
Furthermore, our attempts to simplify the program
  into a form more likely to be found by human analysts
  have only been able to do so by greatly increasing its error.
By properly targetting its search,
  \casio is able to consider a greater number of viable alternatives
  than human analysts are able to,
  and is capable of producing better results.

\todo{Smooth over}

\end{document}
