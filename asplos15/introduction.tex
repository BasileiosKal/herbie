\documentclass[paper.tex]{subfiles}
\begin{document}

\section{Introduction}
\label{sec:intro}

From physical control systems to financial forecasting, many
applications are designed to compute over real numbers.  However,
practical concerns such as latency and memory usage typically force
applications to approximate their results using hardware-supported
floating point operations.  Unfortunately, finite precision leads to
rounding error; for many computations, even miniscule errors can
propagate to produce completely incorrect results.

\todo{examples where this is real bad}

For decades researchers have developed techniques to analyze and
mitigate rounding error, including sophisticated forward and backward
error analysis \todo{cite}, robust multi-precision floating point
representations designed for efficient hardware implementation
\todo{cite}, and numerous program transformations to reduce error
\todo{cite}.  Unfortunately, manually applying these techniques
requires substantial background which few programmers can afford.

In practice, developers have few tools to address rounding error.  A
common strategy is simply increasing precision, for example by
replacing 32-bit floats with 64-bit floats.  While increasing
precision may help by shifting error to lower order bits, the program
can still exhibit unacceptable rounding error.  Furthermore increasing
precision typically doubles floating point memory and latency costs.
Once a programmer has exhausted the available, hardware-supported
precision, she can switch to an arbitrary precision floating point
library.  However, such libraries still have rounding error and impose
prohibitive overhead for many applications. \todo{massage} \todo{back
  these points up with examples from evaluation}

% In practice, many developers have little recourse when they encounter
% unacceptable rounding error.  A common strategy is simply increasing
% precision, for example by replacing 32-bit floats with 64-bit floats.
% This strategy suffers from two primary drawbacks.  First, increasing
% precision typically doubles a program's memory usage and can quadruple
% its latency for floating point.  Second, for naive translations from
% real computations to floating point, increased precision often simply
% shifts error instead of eliminating it \todo{massage}. \todo{back
%   these points up with examples from evaluation}

To address these challenges, we present \casio, a technique to
automatically synthesize hardware-supported floating point code with
low rounding error \todo{massage} from real computations.  Given an
input program $P$ over the reals, \casio performs a heuristic search
for a program $P'$ which is equivalent to $P$ over the reals and
minimizes rounding error.  At each step of the search, \casio
automatically analyzes a candidate program for sources of error,
applies potentially error-reducing rewrites drawn from the numerical
methods literature, and estimates the rounding error of the resulting
program.  By searching through thousands of candidate programs, our
prototype implementation is able to \todo{impressive summary of
  evaluation}.  Furthermore, \casio reduces rounding error
automatically, requiring no training or intervention from the
programmer.

\casio complements several concurrent efforts to improve floating
point computation.  Our technique is parameterized by error estimation
so we can adopt state of the art techniques as they are developed.
STOKE can work as a post processing pass.  Approximate computing
people can something something.

We evaluated \casio on a large suite of microbenchmarks drawn from
numerical methods textbooks, mathematical libraries drawn from the
Web, and recent scientific articles.  Our results show that \casio
\todo{results}.  We also evaluated \casio on two large applications
\todo{do it}.  Finally, \casio has already been found useful by
colleagues in machine learning who were able to significantly reduce
the rounding error of their learning algorithm something something.

In summary, this paper makes three primary contributions:
\begin{enumerate}
\item To the best of our knowledge, the first automated technique to
  mitigate rounding error.
\item The perspective that approximation work should be done vs. real computation.
\item Thorough evaluation and framework for others to build upon.
\end{enumerate}

The rest of the paper is organized as follows.  \cref{sec:background}
provides a brief survey of the relevant background on floating
computation.  \cref{sec:overview} illustrates \casio on a
representative example and describes the high level \casio workflow.
\cref{sec:synthesis} details \casio's heuristic search and error
estimation techniques.  \cref{sec:evaluation} evaluates \casio's
effectiveness across a suite of microbenchmarks, two large
applications, and the effort to build and run \casio.
\cref{sec:relatedwork} surveys the most closely related work and
\cref{sec:futurework} considers future work and concludes.


% Many developers are instilled with a sort of superstitious unease
% concerning floating point.

% Arbitrary precision floating point libraries \todo{cite}.  In the case
% of arbitrary precision floating point libraries, the technique is
% prohibitively expensive.


% In response to the ubiquity and subtlety of floating point
% computation, the numerical methods community has developed a host of
% powerful techniques to analyze and mitigate numerical inaccuracies.
% These approaches rely on a deep understanding of the floating point
% representation, sophisticated error analyses to choose an appropriate
% precision, and optimizations to improve accuracy when additional
% precision is unavailable or too expensive in terms of space or time.
% Unfortunately, while most programmers are instilled with a sort of
% superstitious unease concerning floating point, very few possess the
% background required to manually apply the traditional numerical
% methods techniques that address these challenges.

% To ease the programmer's burden, researchers have developed automated
% techniques to mitigate some challenges in achieving numerical
% accuracy.  The resulting tools not only make the hard won knowledge of
% the numerical methods community more generally accessible to
% non-experts, but also help make experts more effective.  In
% particular, researchers have developed techniques to automatically
% measure error, automatically choose between available
% hardware-supported precisions, and optimize computations to avoid
% expensive operations for precision that is not required.

% In practice, programmers often attempt to avoid numerical inaccuracy
% by simply increasing precision, for example by replacing 32-bit floats
% with 64-bit floats.  However, choosing the right representation is
% difficult.  Speed and space and precision tradeoffs.  Choosing right
% tradeoff requires error analysis.  Several techniques exist to
% automatically analyze floating point precision errors CITE CITE CITE
% and help the developer choose an appropriate precision CITE CITE CITE
% \todo{Precimonious.  Lam from UMD.  lots of other automated error
%   analysis}.  Note that, even when applications can afford to simulate
% arbitrary precision floats in software, these challenges still arise.

% Hardware-supported floating point representations come in coarse,
% exponentially growing sizes.  Recent work from the approximate
% computing community helps avoid paying speed and energy overheads for
% precision the developer may not need. \todo{STOKE. Luis. Martin.}  New
% verification techniques help the programmer ensure they have all the
% bits they think they need \todo{eva}.  Testing for overflow,
% underflow, error \todo{popl13}.

% However, to the best of our knowledge, one traditional task of the
% numerical methods expert remains manual: rewriting the floating
% program to \textit{improve} precision.  This has many benefits.


% ----

% Approximating real computations with floating point is one of the oldest
% forms of approximate computing.  Unfortunately, providing good
% approximations is often incredibly difficult due to the subtleties of how
% rounding error can propagate through a computation.  The problem is that we
% are victims of our success: in practice floating point often works quite
% well, to the degree that most programmers simply think of floating point
% computation as real computation.  Unfortunately, when those computations go
% wrong, they can go horribly, horribly wrong.  Figuring out why requires
% deep understanding of floating point, which is rare and expensive.

% Floating point computation is a well known approach to approximating real
% computation in the bounded context of practical hardware.  Recent advances
% provide techniques to trade-off accuracy for improved speed and power usage
% at the cost of precision.  However, this work is often carried out under
% the assumption that the original floating program is the ``ground truth'',
% that is, that it is already an acceptable approximation of the original
% real computation the programmer had in mind.  Unfortunately, approximating
% real computation in floating point presents many subtle challenges which,
% if not accounted for, can yield arbitrarily large errors.  This limits the
% value of recent advances improving the runtime and power efficiency of such
% computations: the wrong answer is simply arrived at more quickly using
% fewer joules (watts? ugh).

% The field of numerical methods provides a rich set of techniques for
% improving the numerical accuracy and precision of real computations
% approximated in floating point.  Unfortunately, applying these techniques
% has traditionally demanded extensive training and expensive manual analysis
% and optimization of numerical programs.  As more and more scientists write
% numerical programs, this lack of accuracy becomes increasingly troubling.
% These non-expert programmers usually lack the training and time necessary
% to employ numerical methods in their code, and often may not even be aware
% of the subtle challenges this domain presents.

% We present a new tool, \casio, which synthesizes floating point programs
% which better approximate the real computations programmers have in mind.
% Casio works by performing a heuristic search over the space of programs
% equivalent to the input program over the reals.  Each step of the search
% attempts to apply identity transformations over the reals that have been
% shown valuable in the numerical methods literature.  Casio also searches
% for alternate representations of programs that may be challenging for even
% for numerical methods experts to develop, for example, special casing
% variants of the program to input domains with better precision for that
% variant.

% We evaluate \casio on a number of examples drawn from numerical methods
% textbooks, scientific journal articles, and math libraries.  We find that
% \casio is often able to improve the precision of these examples
% considerably.

\end{document}
