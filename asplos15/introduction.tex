\documentclass[paper.tex]{subfiles}
\begin{document}

\section{Introduction}
\label{sec:introduction}

Many applications that depend on floating point arithmetic must
produce accurate results, including fluid dynamics simulators, revenue
predictors for financial forecasting, and statistical analysis
packages used across the sciences.  These applications typically rely
on hardware implementations of floating point to efficiently
approximate computations over real numbers.  Unfortunately, such
approximations introduce rounding error which can lead to unacceptably
\textit{inaccurate} output, where floating point results differ
substantially from their ideal real number results.

% The efficiency provided by hardware implemented floating point
% enables these applications to effectively process large datasets.

Floating point inaccuracy is notoriously difficult to detect and debug
\todo{cite kahan}.  Rounding errors have lead to the retraction of
numerous scientific articles \todo{cite}, legal regulations in finance
\todo{cite}, distorted stock market indices \todo{cite}, and skewed
election results \todo{cite}.

In practice, developers often respond to rounding error by increasing
\textit{precision}, the size of the floating point representation.
For example, a developer might try to shift error to lower order bits
by replacing all 32-bit floats with 64-bit floats.  Unfortunately,
even the largest hardware-supported precision may still exhibit
unacceptable rounding error, and further increases to precision
require simulating arbitrary precision floating point in software,
incurring orders of magnitude slowdown\footnote{Even arbitrary
  precision software floating point can exhibit rounding error, and so
  the developer must still carefully select a precision that provides
  sufficient accuracy.}.

\todo{Pavel here} The numerical methods literature provides numerous
techniques to analyze and mitigate rounding error without increasing
precision. \todo{matrices, loops, stability, interpolation} In
particular, forward and backward error analysis \todo{cite} coupled
with various program transformations \todo{cite} can be applied to
improve accuracy without increasing precision.  Unfortunately,
applying these techniques often requires deep expertise and
substantial manual effort.

We introduce \casio, a first step toward automatically improving
floating point accuracy.  \casio searches for error-reducing program
transformations by combining three novel techniques.  First, \casio
estimates which operations are responsible for rounding error by
carefully sampling floating point inputs and comparing the results of
intermediate operations against accurate results computed in arbitrary
precision software floating point.  Second, based on the error's
characteristics, \casio independently applies various rewrites,
measures their affect on accuracy, and chooses a subset of program
variants to continue improving.  Third, because transformations may
not uniformly improve accuracy across the floating point domain,
\casio detects input regimes where error behavior changes and combines
multiple program variants to reduce overall error.  By combining these
techniques, \casio reduces rounding without requiring training or
manual intervention from the programmer.

We evaluate \casio on benchmarks drawn from numerical methods
textbooks and consider its broader applicability to floating point
expressions extracted from a mathematical library as well as formulas
drawn from recent scientific articles.  Our results demonstrate that
\casio can effectively discover transformations that substantially
reduce error while imposing overhead much less than software floating
point.  By searching through thousands of candidate programs, our
prototype implementation is able to \todo{impressive summary of
  evaluation}.  Furthermore, \casio has already been found useful by
colleagues in machine learning who were able to significantly reduce
the rounding error of their learning algorithm. \todo{better anecdote
  intro}

In summary, \casio contributes three novel techniques to automatically
improve floating point accuracy:
\begin{enumerate}
\item A goal directed search for accuracy improving program
  transformations, guided by techniques that estimate rounding error
  and its causes
\item A technique that iteratively applies local, domain specific
  rewrites, causing traditional error-reducing program transformations
  to fall our naturally
\item Approaches to both detecting regimes where error behavior
  changes and also combining multiple program versions to reduce
  overall error
\end{enumerate}
We evaluate \casio and demonstrate that it is able to effectively
discover accuracy improving transformations.  Furthermore, the \casio
infrastructure provides a foundation for others to build upon and
further explore additional issues around numerical computing such as
additional techniques to mitigate rounding error, improving stability,
matrices, loops, programmer interfaces.

The rest of the paper is organized as follows.  \Cref{sec:background}
provides a brief background on floating point arithmetic.
\Cref{sec:overview} illustrates \casio on a representative example and
describes the high level \casio workflow.  \Cref{sec:synthesis}
details \casio's heuristic search and error estimation techniques.
\Cref{sec:evaluation} evaluates \casio's effectiveness across a suite
of microbenchmarks, two large applications, and the effort to build
and run \casio.  \Cref{sec:relatedwork} surveys the most closely
related work and \Cref{sec:futurework} considers future work and
concludes.


% GRAVEYARD

% Unfortunately, while most programmers are instilled with a sort of
% superstitious unease concerning floating point, very few possess the
% background required to manually apply the traditional numerical
% methods techniques that address these challenges.

% Many developers are instilled with a sort of superstitious unease
% concerning floating point.

% not only make the hard won knowledge of the numerical methods
% community more generally accessible to non-experts, but also help
% make experts more effective

% \casio complements several recent efforts to improve overall floating
% point computation.  Our technique is parameterized by error estimation
% so we can adopt state of the art techniques as they are developed.
% STOKE can work as a post processing pass.

% In particular, researchers have developed techniques to
% automatically measure error, automatically choose between available
% hardware-supported precisions, and optimize computations to avoid
% expensive operations for precision that is not required.

% However, this work is often carried out under the assumption that
% the original floating program is the ``ground truth'', that is, that
% it is already an acceptable approximation of the original real
% computation the programmer had in mind.  Unfortunately,
% approximating real computation in floating point presents many
% subtle challenges which, if not accounted for, can yield arbitrarily
% large errors.  This limits the value of recent advances improving
% the runtime and power efficiency of such computations: the wrong
% answer is simply arrived at more quickly using fewer joules (watts?
% ugh).

% Casio works by performing a heuristic search over the space of
% programs equivalent to the input program over the reals.  Each step
% of the search attempts to apply identity transformations over the
% reals that have been shown valuable in the numerical methods
% literature.  Casio also searches for alternate representations of
% programs that may be challenging for even for numerical methods
% experts to develop, for example, special casing variants of the
% program to input domains with better precision for that variant.

\end{document}