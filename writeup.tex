\documentclass{article}
\usepackage{graphicx}
\usepackage{amsmath}
\usepackage{booktabs}

\title{Casio Writeup (Working Title)}

\newcommand{\insetarrow}[1]{\stackrel{#1}{\Rightarrow}}
\newcommand{\lnexp}[1]{\ln{\left(e^{#1}\right)}}
\newcommand{\sqrsqrt}[1]{\sqrt{#1}^2}

\begin{document}

\maketitle

\section{Casio}

In the real world, 
we find ourselves dealing a lot with formulas 
on floating point numbers. 
And as programmers, 
we know that these floating point calculations 
are inherently inprecise. 
Programs can be rewritten by an expert to reduce their floating point error, 
but such numerical experts are often out of the budget of teams wishing to use these rewrites, 
and floating point calculations can be very large and complex. 
Casio aims to automate this process of rewriting floating point expressions 
to reduce their total error. 
From here on I will refer to the floating point expressions 
which casio operates on 
as ``programs.''

Casio on a whole operates through a set of ``stategies'', 
each of which operates by making simple rewrites according to a set of rewrite rules. 
Rewrite rules are mathamatically sound transformations across the reals, 
such as $a - b \to \frac{a^2 - b^2}{a + b}$, 
or $x \to e^{\ln{x}}$
\footnote{Special cases such as when $a = -b$ or $x < 0$ are accounted for seperately.}.
Each strategy attempts to apply the best rewrites 
and reduce program error. 
Strategies include algebriac simplification of the program, 
searching through the expression tree to find the node with the greatest local error 
and attempting to use rewrite rules to destruct that node, 
and, if all else fails, 
brute force searching through the space of equivilent programs 
using the total error of each alternative program 
as a heurestic to guide the search.

Over the next two sections,
we'll discuss two features of the Casio search
which help to streamline the search and it's results.

\section{Red Elimination}

Each strategy employed by casio 
has it's own merits and pitfalls, 
and each strategy has the potential 
to produce the optimal set of changes on certain programs, 
while producing non-optimal changes on others. 
Sometimes, a strategy will pick a change 
that does not contribute to the reduction of error, 
before picking a change that does. 
In the brute force search this was initially quite common, 
with many results differing from the optimal result 
by a $\lnexp{x}$ inserted in the place of $x$ 
somewhere in the program. 
Even when we apply changes based on which will destruct a node 
that is evaluated to have the most local error, 
that local error could jump to somewhere else in the tree, 
and when that node is optimized, 
it might turn out that the destruction of the first node was unneccessary. 
Thus, it is advantages for Casio 
to implement what we call Red Elimination, 
the elimination of changes, after the fact, 
that did not contribute to the reduction of errors.

Red Elimination works by classifying changes made to a program by Casio in one of three classes: 
red, orange, and green. 
Green changes are the simplest to define and identify: 
a given change is green if it immediately reduces the total error in the program. 
That is, if we measure the error of a program to be some $\epsilon$, 
then we make a change, 
and then we measure the error of the program to be $\epsilon'$, 
where $\epsilon' < \epsilon$, 
then the change we made was a green change. 
Each version of the program carries with it 
it's total error across the sample points, 
so detecting green changes is a simple operation. 
But changes do not exist in isolation. 
Many times, the program is initially not in a form 
that makes the eventual green changes immediately possible. 
Changes that do not themselves reduce error, 
but help transform the program into a form 
that allows for one or more green changes, 
are classified as orange changes. 
Sometimes there are many orange changes needed 
before casio can make a green change, 
and determining which of many potential orange changes 
to make is one of the hardest parts of Casio's operation.

Finally, there exist red changes, 
changes which Casio makes to a program 
that turn out to be completely unecessary 
for the creation of the eventual green changes. 
Red Elimination attempts to identify these red changes, 
and once identified, 
remove them entirely from the current version of the program. 
After each green change is found, 
Casio uses Red Elimination to eliminate red changes from the change history, 
so that at any given invocation of Red Elimination, 
red changes only exist between the most recent change, 
a green change,
and the green change that came before it.

To eliminate these red changes, 
Red Elimination develops the notion of translating changes through one another. 
Simply put, 
given a program such as $(1 + x) - x$, 
casio might make the changes

\begin{align*}
(1 + x) - x
&\to (1 + \lnexp{x}) - x\\
&\to 1 + (\lnexp{x} - x)
\end{align*}

The second change, 
the reassociation, is a green change, 
since it causes the $x$'s to immediately cancel, 
and ensure that the program always returns one, 
instead of the zero that the original form might return 
given a large enough $x$. 
If we name the first change $a$, 
and the second change $b$, 
we might write this as 

\begin{align*}
(1 + x) - x&\insetarrow{a} (1 + \lnexp{x}) - x\\
&\insetarrow{b} 1 + (\lnexp{x} - x)
\end{align*}

or 

\[(1 + x) - x \insetarrow{ab} 1 + (\lnexp{x} - x)\]

If, as in this case, 
the changes can be translated through one another, 
there also exist changes a' and b', 
such that 

\[(1 + x) - x \insetarrow{b'a'} 1 + (\lnexp{x} - x)\]

or

\begin{align*}
(1 + x) - x &\insetarrow{b'} 1 + (x - x) \\
&\insetarrow{a'} 1 + \lnexp{x} - x)
\end{align*}


It should be clear from inspection that, in fact, 
$b'$ is a valid green change on the initial program, 
without having to involve $a'$ or $a$ at all. 
In Red Elimination terminology, 
a was a red change, 
and we were able to eliminate it 
by translating it through the green change, 
and then simply cutting it off. 
While many cases that Red Elimination hopes to handle 
are not this simple, 
the principle remains the same: 
If we can translate a change 
forward in the change history 
until it is past the most recent green change, 
then that change is a red change, 
and we can eliminate it by simply cutting it off 
once it has been translated through. 
If we can not translate the change through the green change, 
then that change is an orange change, 
and it is necessary for the existence of the green change.

Red Elimination thus consists of two parts: 
an algorithm which, given two changes, 
attempts to translate them through each other, 
and an algorithm which uses this translation capability 
to attempt to move all changes since the preceding green change 
forward in the change history 
through the current green change. 
If the former cannot translate the changes, 
then it returns false, 
and the latter knows that these changes block one another.

\subsection{(translate): Attempting to Translate Change Pairs}

Before we can discuss the nitty-gritty of translating changes through each other, 
it is important to establish the representation that Casio uses for changes 
and the definition of equivilent changes, 
so that we know that a and a' are equivilent. 
To Casio, changes are represented as a location in the expression tree, 
a ``rule,'' or mathematically sound transformation across the reals, 
and bindings for this rule. 
Rules have input and output patterns, 
such as $a + (b + c) \to (a + b) + c$,
and when they are applied, 
any expressions can be substituted for the rule variables 
($a$, $b$, and $c$ in this case). 
For instance, 
the above rule could be applied 
to $\frac{x}{y} + (1 + -y)$ 
to make $(\frac{x}{y} + 1) + -y$, 
with the substitutions $a = \frac{x}{y}$, $b = 1$ and $c = -y$. 
These substitutions are called ``bindings'', 
and they are the third part of the change structure.

For two changes to be considered equivilent, 
they must have the same rule, 
and share either their location, 
or their bindings. 
If the changes do not share their locations, 
then the transformation from the original changes location 
to the new changes location must be a reasonable translation 
of the location across the intervening change. 
This definition of equivilence, 
while at first a little abstract, 
becomes more concrete in the implementation. 
This definition of equivilence is created to ensure 
that any change equivilent to a green change 
is also itself a green change, 
so that any valid translation of a green change 
through another change 
will not strip the ``greenness'' from the change.

In order to more fully explore what it means
to translate two changes through each other,
let's look at three examples
of change pairs that casio might try to translate through each other:

\begin{align*}
\label{eq:changes}
y + x &\insetarrow{a} \lnexp{y} + x \\
&\insetarrow{b} \lnexp{y} + \sqrsqrt{x}\\ \\
y + x &\insetarrow{a} \lnexp{y} + x \\
&\insetarrow{b} \sqrsqrt{\lnexp{y}} + x\\ \\
(1 + x) - x &\insetarrow{a} 1 + (x - x) \\
&\insetarrow{b} 1 + 0
\end{align*}

The first set of changes appears trivial to translate;
since the first change only operates on the $y$,
and the second change only operates on the $x$.
In fact, this change \emph{is} easy for casio to translate.
We call changes like this first one
independent changes,
since they do not interfere with each other,
and can easily be translated.
Once translated, this first change pair looks like:

\begin{align*}
y + x &\insetarrow{a} y + \sqrsqrt{x}\\
&\insetarrow{b} \lnexp{y} + \sqrsqrt{x}
\end{align*}

In more technical terms,
independent changes are changes whose locations diverge
so that they operate on different parts of the expression tree.
Translation of two independent changes requires
no modification of either change structure,
instead, 
Casio simply moves the change objects
past each other in the change history.

The second pair of example changes
are more difficult for Casio to translate,
since the second change, change $b$,
operates on top of the first change, change $a$.
Since $b$ is taking the result of the $a$ as it's input,
at first it seems that we have no guarantee
that $b$ would work without $a$.
In fact, we \emph{do} have a guarantee 
that $b$ will work without $a$,
we just need to look at the rules and bindings of each change.
The $a$ has rule $i\to\lnexp{i}$
and bindings $i=y$,
and $b$ has rule $i\to\sqrsqrt{i}$
and bindings $i=\lnexp{y}$.
We can see here that $a$ is actually completely contained
within the bindings of $b$.
We call change pairs like these binding-contained changes.
While binding-contained changes are harder to translate than independent changes,
it is still possible for Casio to translate them,
but we must this time modify the change structure.
In order to translate binding-contained changes,
the inner change must have it's location translated appropriately,
and the outer change must have it's bindings rebound.
In our example binding-contained changes:

\begin{align*}
y + x &\insetarrow{a} \lnexp{y} + x \\
&\insetarrow{b} \sqrsqrt{\lnexp{y}} + x
\end{align*}

We translate the \emph{location} of $a$,
the inner change,
from ``The first argument of the addition''
to ``The argument to square-root in square in the first argument of the addition'',
and we rebind the \emph{bindings} of $b$,
from $i=\lnexp{y}$
to $i=y$.
Now, we can make these changes in the reverse order,
first $b'$ (the translated version of $b$),
then $a'$ (the translated version of $a$).

\begin{align*}
y + x &\insetarrow{b'} \sqrsqrt{y} + x\\
&\insetarrow{a'} \sqrsqrt{\lnexp{y}} + x
\end{align*}

Finally, consider the third pair of example changes:

\begin{align*}
(1 + x) - x &\insetarrow{a} 1 + (x - x) \\
&\insetarrow{b} 1 + 0
\end{align*}

We can see right away that these changes are not independent:
the first change operates on the whole expression,
and the second change operates on the second argument to the addition.
The rule and bindings of $a$ are $\{(i + j) - k \to i + (j - k)$, $i=1 j=x k=x\}$,
and the rule and bindings of $b$ are $\{i-i\to0$, $i=x\}$,
so this change pair doesn't appear to be binding contained either.
In fact, these changes cannot be translated through each other by Casio.
In general, any change pair that is neither independent nor binding-contained
is what we call \emph{dependent},
and the changes can't be translated through each other.
When (translate) is called on dependent changes, it returns false.

\subsection{(remove-red): Using Translation to Eliminate Red Changes}

Once we have translation of pairs of changes established, 
the question becomes how do we use this pairwise translation 
to eliminate as many changes 
as possible from our change history? 
Instead of thinking of moving changes 
forward in the change history 
to surpass the green change, 
the problem becomes simpler 
if we think of it the other way around: 
attempting to move green changes 
as far backward 
in the change history as possible. 
The algorithm we use for this problem 
is fairly simple in principle: 
First, attempt to move the green change back one change. 
If this is sucessful, 
throw out the change that the green change moved through, 
and recurse on the new change history. 
If this translation is unsucessful, 
then the green change is directly dependent 
on the change immediately before it. 
As a result, 
this change is an orange change, 
and will (besides for a few special cases) 
remain within the change history. 
But, there may be other 
red changes that occur before this orange change, 
so we then recurse on the orange change, 
trying to move it back in the change history 
as much as possible. 
When this recursion returns, 
we first test to see if the orange change 
was moved back at all, 
and if so, 
we recurse on the current change history 
to continue to move the green change back. 
The recursive call is somewhat different however, 
when it operates on a non-green change (any orange change), 
as instead of throwing away changes 
that we move past, 
we must keep them in the change history 
as they are still in front of the green change.

A simple visual example 
might help to further illustrate this process.
\begin{itemize}
\item We start with a change history 
with some of each type of change. 
For convenience, 
only the part of the history 
between the head green change 
and the previous green change 
is shown, 
as the algorithm only operates 
on that part of the 
change history. 
Red changes are to be represented as $r_n$, 
orange changes as $o_n$, 
and green changes as $G_n$. 
A star above the change
that it is the change we are currently operating on.

\[G_0\to r_0\to r_1\to o_0
\to r_2\to o_1\to o_2\to r_3
\to r_4\to o_3\to r_5 \to \stackrel{*}{G_1}\]
\item The second to last change is red,
and thus can be translated
through the green change
and thrown away.

\[G_0\to r_0\to r_1\to o_0
\to r_2\to o_1\to o_2\to r_3
\to r_4\to o_3\to \stackrel{*}{G_1}\]

\item The next change is orange,
so we cannot translate the green change
through it.
However, we can attempt
to translate the orange change
backwards in the change history
to further eliminate reds.

\[G_0\to r_0\to r_1\to o_0
\to r_2\to o_1\to o_2\to r_3
\to r_4\to \stackrel{*}{o_3}\to G_1\]

\item The change before that orange is red, 
so we can move through it.

\[G_0\to r_0\to r_1\to o_0
\to r_2\to o_1\to o_2\to r_3
\to \stackrel{*}{o_3}\to r_4 \to G_1\]

\item Still recursing on the orange change, 
we can move it back another red change.

\[G_0\to r_0\to r_1\to o_0
\to r_2\to o_1\to o_2 \to
\stackrel{*}{o_3}\to r_3 \to r_4 \to G_1\]

\item Now, we've hit a change we can't translate through, 
so we recurse on the blocking change.

\[G_0\to r_0\to r_1\to o_0
\to r_2\to o_1\to \stackrel{*}{o_2}\to o_3
\to r_3 \to r_4 \to G_1\]

\item This change is also blocked, 
so we recurse again.

\[G_0\to r_0\to r_1\to o_0
\to r_2\to \stackrel{*}{o_1}\to o_2 \to o_3
\to r_3 \to r_4 \to G_1\]

\item This change can be moved back, 
so we do so.

\[G_0\to r_0\to r_1\to o_0
\to \stackrel{*}{o_1}\to r_2 \to o_2 \to o_3
\to r_3 \to r_4 \to G_1\]

\item And now it's blocked, 
so we once again recurse.

\[G_0\to r_0\to r_1\to \stackrel{*}{o_0}
\to o_1\to r_2 \to o_2 \to o_3
\to r_3 \to r_4 \to G_1\]

\item This change can be moved back twice 
(for simplicities sake, 
I will begin showing multiple trivial steps as one step)

\[G_0\to \stackrel{*}{o_0}\to r_0\to r_1
\to o_1\to r_2 \to o_2 \to o_3
\to r_3\to r_4 \to G_1\]

\item We've hit the beginning, 
so we return to the change that called us.

\[G_0\to o_0\to r_0\to r_1
\to \stackrel{*}{o_1}\to r_2 \to o_2 \to o_3
\to r_3\to r_4 \to G_1\]

\item We can now see that the change before us 
is not the change that was before us 
before the recursive call, 
so we attempt to move backwards, 
and can succeed for two iterations.

\[G_0\to o_0\to \stackrel{*}{o_1} r_0
\to r_1 \to r_2 \to o_2 \to o_3
\to r_3\to r_4 \to G_1\]

\item This change will recurse again, 
but the recursion will immediately return 
with the same previous change 
as we had before, 
so this change is done 
and returns to it's calling change.

\[G_0\to o_0\to o_1\to r_0
\to r_1 \to r_2 \to \stackrel{*}{o_2}\to o_3
\to r_3\to r_4 \to G_1\]

\item This change moves back three steps 
before finishing in the same way as it's predecessor

\begin{align*}
&G_0\to o_0\to o_1\to o_2
\to r_0\to r_1 \to r_2 \to \stackrel{*}{o_3}\to
r_3\to r_4 \to G_1\\
&G_0\to o_0\to o_1\to o_2
\to o_3 \to r_0 \to r_1\to r_2
\to r_3 \to r_4 \to \stackrel{*}{G_1}\\
&G_0\to o_0\to o_1\to o_2
\to o_3 \to \stackrel{*}{G_1}
\end{align*}

\item And so the process finishes, 
with all red changes eliminated.
\end{itemize}

In short, 
Red Elimination allows us to eliminate
red changes from our change history 
and contributes to the final outputted 
program having the least error possible 
while remaining as simple 
and close to the input program 
as possible.

\section{Simplification}

Suppose we give Casio the program $\sqrt{x+1} - \sqrt{x}$. 
We know that there is an improved program, 
$1/\sqrt{x+1} + \sqrt{x}$, 
which is equivilent over the reals, 
and as such is the ideal output 
for Casio given the initial program. 
The trick is teaching Casio 
to find this program. 
Given our existing rewrite rules, 
the ideal transformation is 

\begin{align*}
\sqrt{x + 1} - \sqrt{x} &\to \frac{\sqrsqrt{x + 1} - \sqrsqrt{x}}{\sqrt{x + 1} + \sqrt{x}}\\
&\to \frac{(x + 1) - \sqrsqrt{x}}{\sqrt{x + 1} + \sqrt{x}}\\
&\to \frac{(x + 1) - x}{\sqrt{x + 1} + \sqrt{x}}\\
&\to \frac{(1 + x) - x}{\sqrt{x + 1} + \sqrt{x}}\\
&\to \frac{1 + (x - x)}{\sqrt{x + 1} + \sqrt{x}}\\
&\to \frac{1 + 0}{\sqrt{x + 1} + \sqrt{x}}\\
&\to \frac{1}{\sqrt{x + 1} + \sqrt{x}}
\end{align*}

However, this is a lot of steps, 
and the likelyhood that these steps 
would happen through brute force 
is pretty small, 
given the number of possible 
changes to search. 
Our focused search strategy, 
where we select the node 
with the most error 
and attempt to destruct it, 
fares slightly better, 
as it immediately sees that the subtraction is the problem 
and returns the first step as one 
of a pretty small set of possible steps. 
Unfortunately, 
from there it again sees the subtraction as problematic, 
and attempts to destruct it again. 
The problem is slightly easier than it might at first appear, 
because once we reassociate the x's to cancel, 
the error is immediately reduced 
and a green change is achieved, 
so the last two steps are not strictly necessary. 
But, it would still be nice to have those last two steps, 
as they result in a more readable program 
as well as a program that is cheaper to execute. 
It is here that a controlled form 
of simplification is beneficial.

We can look at getting the ideal output
 of this program 
in two parts. 
The first part destructs the minus, 
turning the program into

\[\frac{\sqrsqrt{x + 1} - \sqrsqrt{x}}{\sqrt{x + 1} + \sqrt{x}}\]

The second part is all about simplifying 
the top half of that division 
down to a simple constant, 
$1$.

\[\frac{1}{\sqrt{x+1}+\sqrt{x}}\]

We can see 
that a quick simplification 
speeds up the process 
and makes us a lot more likely 
to achieve our goal program. 

Most of Casio's strategies
operate by making incremental changes 
to programs using our rewrite rules. 
This approach works well 
because it allows us to reevaluate the program 
at every step 
and make sure that we are guiding our search 
towards our goal. 
There is one strategy, however, 
that doesn't make incremental changes, 
but instead tries to make 
large scale transformation of programs. 
This strategy, simplification, 
doesn't need to evaluate 
the program at every step 
because instead of being guided 
by the larger abstract goal 
to reduce error, 
the simplification process only seeks 
to make a program as simple as possible, 
under the assumption that 
simpler is (almost) always better.

Simplifying, 
as it currently stands 
operates on the basic principle 
that making a change to the 
simplest possible version of a program, 
and then simplifying the resulting program 
in such a way that 
it does not destruct the change, 
is the best way to determine 
whether a change is beneficial 
to the program.
After each change made by one of the other strategies,
Casio attempts to simplify the resulting expression,
and keeps the simplified version
if it has better error than the unsimplified version.

\subsection{Simplification Locations}

In the above example,
smart simplification of the program after the first step,
\begin{align*}
\frac{\sqrsqrt{x+1} - \sqrsqrt{x}}{\sqrt{x+1} + \sqrt{x}}
\to \frac{1}{\sqrt{x+1}+\sqrt{x}}
\end{align*}
is extremely beneficial to our search.
However, note that the initial program, 
$\sqrt{x+1}-\sqrt{x}$,
is actually even simpler than it's error-reduced form,
$\frac{1}{\sqrt{x+1}+\sqrt{x}}$.
Therefore, a perfect simplification
of the program after the first step
would actually revert
the improvements we made to the program.
Instead, we would like our simplifier
to only simplify
the numerator of the division.
We generalize that for any given change, 
there are certain locations 
for which it would be beneficial to simplify. 
These locations are determined 
by the rule which the change applies. 
The change

\[\sqrt{x + 1} - \sqrt{x} \to \frac{\sqrsqrt{x + 1} - \sqrsqrt{x}}{\sqrt{x + 1} + \sqrt{x}}\]

applies the rule 
$a - b \to  \frac{a^2 - b^2}{a + b}$. 
In this change, 
the division is what is important 
and should be preserved, 
so the simplification locations 
for the rule are the numerator 
and denominator of the division. 
Attempting to simplify at both these locations, 
for any invocation of this rule, 
could possibly create green changes 
(as seen in motivating example), 
and furthermore, 
we assert that there are no simplifications 
that would cause a green change 
which will not be found 
by simplifying at these locations.

Other rules, like reassociation,
also have simplification locations.
For the rule $(a + b) - c \to a + (b - c)$,
we simplify at the second argument of the addition,
$(b - c)$. 
While it may seem like the simplifier
should try the whole expression,
we don't want to risk undoing the reassociation.
One might also think that the simplifier
should also simplify the $a$,
but Casio attempts to simplify after every step,
so we can guarantee that the expression
before each change is in ``beneficial simplified form,''
meaning that if it would have been beneficial
to simplify the expression,
then the expression was simplified. 
Therefore, we know that the input
$(a + b) - c$ is in ``beneficial simplified form,''
and by extension, each of it's subexpressions is in
``beneficial simplified form,''
so attempting to simplify $a$ would do nothing.

For the same reason, 
many of our rewrite rules
don't actually have any simplification locations.
Rules like $\lnexp{x}\to x$
don't have simplification locations,
because if $\lnexp{x}$ is in
``beneficial simplified form,''
then $x$ is also in ``beneficial simplified form,''
so there is nothing to simplify on the output.
Rules like $x\to\lnexp{x}$
also don't have simplification locations,
because simplifying the whole thing
would simply eliminate the whole change,
and simplifying anything smaller
would do nothing.

\subsection{(simplify-expression): Quickly Simplifying Most Expressions}

Once we know where to simplify, 
and we have systems in place 
to throw out our simplifications 
if they don't produce a green change, 
all that's left is to simplify the given subexpressions. 
Simplification is accomplished 
by a recursive function 
which works from the leaf nodes 
of the expression up, 
simplifying as it goes. 
Every time it reaches a node, it:

\begin{enumerate}
\item Attempts to apply all reduction rules. 
Reduction rules are defined as all rules 
in our rewrite-rules 
that take the form $f(x) \to x$, 
such as $\lnexp{x} \to x$, 
or $\sqrt{x^2} \to x$.

\item Attempts to precompute functions of constants. 
If the expression 
is of the form $f(c_1, c_2, c_3,..., c_n)$, 
where $c_1$--$c_n$ are constants, 
then the value of the function invocation 
is known at compile time. 
Therefore, we can further simplify the expression 
by calculating the value of the function 
in arbitrary precision 
(the details of which are out of the scope of this section), 
and then substituting that value 
in for the function invocation 
(unless the value is a non-number, 
such as NaN, 
in which case we leave the function invocation 
as it is.)

\item Attempts to apply one 
of a set of simplification rules 
depending on the form of the expression. 
These simplification rules 
all act to attempt to cancel terms and factors 
where cancellable terms and factors are present, 
and to change the expression into a canonical form 
so that more cancelation may happen 
farther up the tree. 
Two of the simplification rules 
also include a recursive call to simplify, 
to work subtractions and divisions down the tree.
\end{enumerate}

We demonstrate the effectiveness of this algorithm 
with a simple example:

\begin{enumerate}
\item Given the expression
$((\sqrt{x} + \sqrsqrt{x}) + -((x + 1) - 1))^2$,
\item We first simplify the leaf nodes,
$((\sqrt{x} + x) -x)^2$,
\item then move one level up, 
$\sqrt{x}^2$,
\item and finally simplify the top level,
$x$.
\end{enumerate}

With our simplification algorithm, 
we can rewrite several classes of program
in fewer iterations.

\section{Conclusions}
To test the effectiveness 
of both the Red Elimination 
and the Simplication features,
we developed a method of aggregating
the results of sucessive calls to improve.
Each data point is the average accuracy improvement
of one hundred calls to improve
with the indicated program
and number of max iterations,
as well as the indicated changes to improve.
Each call to improve has a timeout of 
$1500*1.2^{\verb|max-iters|}$ milliseconds,
where \verb|max-iters| is the maximum number of iterations
that the call to improve was given.
When a call times out,
it is discluded from the average.
Test runs with excessive timeouts are invalid,
and are redone.

\newcommand{\ra}[1]{\renewcommand{\arraystretch}{#1}}
\begin{table*}\centering
\ra{0.8}
\begin{tabular}{@{}rrrrcrrrcrrr@{}}
\toprule&\multicolumn{2}{c}{program-a} &\multicolumn{2}{c}{program-b} &\\
\cmidrule{2-3}\cmidrule{4-5}
& Simplification & Without Simplification & Simplification & Without Simplification\\ \midrule
$\verb|max-iters|=1$\\
$c$ & 0 & 0 & 463.686 & 105.51\\
$\verb|max-iters|=2$\\
$c$ & 0 & 0 & 463.686 & 105.51\\
$\verb|max-iters|=3$\\
$c$ & 0 & 0 & 463.686 & 105.51\\
$\verb|max-iters|=4$\\
$c$ & 0 & 0 & 463.686 & 105.51\\
$\verb|max-iters|=5$\\
$c$ & 0 & 0 & 463.686 & 105.51\\
$\verb|max-iters|=6$\\
$c$ & 0 & 0 & 463.686 & 105.51\\
$\verb|max-iters|=7$\\
$c$ & 0 & 0 & 463.686 & 105.51\\
$\verb|max-iters|=8$\\
$c$ & 0 & 0 & 463.686 & 105.51\\
$\verb|max-iters|=9$\\
$c$ & 0 & 0 & 463.686 & 105.51\\
$\verb|max-iters|=10$\\
$c$ & 0 & 0 & 463.686 & 105.51\\
\bottomrule
\end{tabular}
\end{table*}
\end{document}
